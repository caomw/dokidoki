\chapter{Get Started}

\section{Run sample games}

Inside \textbf{sample} folder in dokidoki, there are sample games on multiple platforms. You could run it to see what games dokidoki could build.

\section{Develop my first game}

Here is the step-to-step tutorial to develop a AVG game based on dikidoki. To start development of your game, firstly create a new folder named MyFirstdokidokiProject (In my case C:/\_Projects/MyFirstdokidokiProject).

\subsection{Write my first dokiScript}

Open your Atom, new file, and save it to disk (../MyFirstdokidokiProject/scripts/MyFirstScript.dks). \textbf{scripts} folder name could be any other name, but it is better not to contain space.

Then start to write your dokidoki scripts. Write the script just the same as below. To learn more detail about scripts writing, you should take a look at Chapter 5. Though you don't know much about dokiScript's grammar now. You could still understand the script easily, for that this script language is designed to be intuitive. 

\begin{lstlisting}
world
	background src=ba0 transition=instant;
	bgm src=bg0;
	>There are someone stand beside the school gate.
	>She looks anxious.

Mary
	role type=character name="Mary";
	posture src=mary000;
	voice src=mary000 >It is almost the time.

world
	>I find she is there.
	>And I walk towards her slowly.
	>However, she does not notice me.

I
	role type=player name="Jack";
	voice >Hello, Mary.

world
	>I said to her.
	>Heared what I said, she turns happy face immediately.

Mary
	voice src=mary001 >I am happy to see you again, Jack.
\end{lstlisting}

After wrote this above dokiScript code, save it and close Atom.

\subsection{Create my first dokiUnity project}

\subsubsection{Setup dokiUnity template}

Then, create dokiUnity project to combine all your game resources. Open Unity, create a new project named MyFirstdokiUnityProject. Then import custom package from \textbf{dokiUnity} folder named \textbf{dokiUnityTemplate.unitypackage}. \textbf{dokiUnityTemplate.unitypackage} contains dokidoki's Unity parts and folder structure to arrange all game resources. Each folder has a \textbf{readme.md} file to explain about what files should be put here. Up to now, you have setup the dokiUnity template.

\subsubsection{Compile my dokiScript files}

Your dokiScript files have extension \textbf{.dks}. You should compile them using \textbf{dokiScriptCompiler} before you put them into dokiUnity project.

Open \textbf{dokiScriptCompiler} sub-project in folder \textbf{/dokidoki/src/dokiScriptCompiler} with \textbf{Xamarin}. To compile all your dokidoki scripts, on the menu bar, click \textbf{Run}---\textbf{Run With}---\textbf{Custom Parameters}, insert your scripts folder absolute path into \textbf{Arguments} row, and then click \textbf{Execute}. If your scripts don't contain any syntax errors, \textbf{dokiScriptCompiler} would generate compiled dokiScripts with \textbf{.txt} extension for each script files.

\subsubsection{Put all game resources into responding folders}

With dokiUnity template, you could start to put your game resources into responding folders. See the dokiScript(not compiled one) you write, you could find that \textbf{src} key parameters have values like \textbf{ba0}, \textbf{bg0} and \textbf{mary000}. These values are names of your responding game resource files, but they are all without extensions. 
\begin{itemize}
	\item You should have a background image file named \textbf{ba0} with proper extension (maybe ba0.png) and put it into folder \textbf{/Assets/Resources/World/Backgrounds}. 
	\item You should have a bgm audio file named \textbf{bg0} with proper extension and put it into folder \textbf{/Assets/Resources/World/Bgms}. 
	\item You should have a posture image file named \textbf{mary000} with proper extension and put it into folder \textbf{/Assets/Resources/Characters/Postures}.
	\item  You should have a voice audio file named \textbf{mary000} with proper extension and another voice audio file named \textbf{mary001} with proper extension, and put them into folder \textbf{/Assets/Resources/Characters/Voices}.
\end{itemize}

And most importantly, your compiled dokiScript file (script with \textbf{.txt} extension) is also a kind of game resources. You should put all of them into folder \textbf{/Assets/Resources/DokiScripts}.

\subsection{Build my first game}

\subsubsection{Have a run for my game}

Now, you almost complete your game. Once you put all your game resources into responding folders. Use Unity to run your game, click \textbf{play} button on the \textbf{Toolbar} and your game would start. It is the most exciting moment for me too.

\subsubsection{Build for Windows, Mac OS}

Nothing different from normal Unity games build. On the \textbf{Topbar} click \textbf{File}---\textbf{Build Settings}. You would see a pop-up window, on the left option for platforms, choose \textbf{PC, Mac, Linux Standalone}, then on the right side choose \textbf{windows}, finally click \textbf{Build} choose a folder to export the runable game (in my case \textbf{/MyFirstdokidokiProject/windows}). Then, Unity would start to build your game, after it finished. Check the export folder you chose before, you would see your game is already there. Wait for what? Play it!

\subsubsection{Build for Android}

In order to build for mobile platforms like Android, the extra work you need to do is to move all your video resources from original folder \textbf{/Assets/Resources/World/Videos} into new folder \textbf{/Assets/StreamingAssets/World/Videos}. The other part is similar to \textbf{Build for Windows, Mac OS}.
