\chapter{Get Started}

\section{Run sample games}

Inside \textbf{sample} folder in dokidoki, there are sample games on multiple platforms. You could run it to see what games dokidoki could build.

\section{Develop my first game}

Here is the step-to-step tutorial to develop a AVG game based on dikidoki. To start development of your game, firstly create a new folder named MyFirstdokidokiProject (In my case C:/\_Projects/MyFirstdokidokiProject).

\subsection{Write my first dokiScript}

Open your Atom, new file, and save it to disk (../MyFirstdokidokiProject/scripts/MyFirstScript.dks). \textbf{scripts} folder name could be any other name, but it is better not to contain space.

Then start to write your dokidoki scripts. Write the script just the same as below. To learn more detail about scripts writing, you should take a look at Chapter 5. Though you don't know much about dokiScript's grammar now. You could still understand the script easily, for that this script language is designed to be intuitive. 

\begin{lstlisting}
world
	background src=ba0 transition=instant;
	bgm src=bg0;
	>There are someone stand beside the school gate.
	>She looks anxious.

Mary
	role type=character name="Mary";
	posture src=po0;
	voice src=mary000 >It is almost the time.

world
	>I find she is there.
	>And I walk towards her slowly.
	>However, she does not notice me.

I
	role type=player name="Jack";
	voice >Hello, Mary.

world
	>I said to her.
	>Heared what I said, she turns happy face immediately.

Mary
	>I am happy to see you again, Jack.
\end{lstlisting}

After wrote this above dokiScript code, save it and close Atom.

\subsection{Create my first dokiUnity project}



\subsection{Build my first game}