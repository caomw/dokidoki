\documentclass[a4paper]{book}

\usepackage{layouts}
\usepackage{subfiles}
\usepackage{listings}
\usepackage{color}
\usepackage{longtable}
\usepackage[utf8]{inputenc}
\usepackage{hyperref}

\usepackage{array}
\newcolumntype{L}[1]{>{\raggedright\let\newline\\\arraybackslash\hspace{0pt}}m{#1}}
\newcolumntype{C}[1]{>{\centering\let\newline\\\arraybackslash\hspace{0pt}}m{#1}}
\newcolumntype{R}[1]{>{\raggedleft\let\newline\\\arraybackslash\hspace{0pt}}m{#1}}


\definecolor{mygreen}{rgb}{0,0.6,0}
\definecolor{mygray}{rgb}{0.5,0.5,0.5}
\definecolor{mymauve}{rgb}{0.58,0,0.82}

\lstset{ %
  backgroundcolor=\color{white},   % choose the background color; you must add \usepackage{color} or \usepackage{xcolor}
  basicstyle=\footnotesize,        % the size of the fonts that are used for the code
  breakatwhitespace=false,         % sets if automatic breaks should only happen at whitespace
  breaklines=true,                 % sets automatic line breaking
  captionpos=b,                    % sets the caption-position to bottom
  commentstyle=\color{mygreen},    % comment style
  deletekeywords={...},            % if you want to delete keywords from the given language
  escapeinside={\%*}{*)},          % if you want to add LaTeX within your code
  extendedchars=true,              % lets you use non-ASCII characters; for 8-bits encodings only, does not work with UTF-8
  frame=single,	                   % adds a frame around the code
  keepspaces=true,                 % keeps spaces in text, useful for keeping indentation of code (possibly needs columns=flexible)
  keywordstyle=\color{blue},       % keyword style
  language=xml,                 % the language of the code
  otherkeywords={*,...},           % if you want to add more keywords to the set
  numbers=none,                    % where to put the line-numbers; possible values are (none, left, right)
  numbersep=5pt,                   % how far the line-numbers are from the code
  numberstyle=\tiny\color{mygray}, % the style that is used for the line-numbers
  rulecolor=\color{black},         % if not set, the frame-color may be changed on line-breaks within not-black text (e.g. comments (green here))
  showspaces=false,                % show spaces everywhere adding particular underscores; it overrides 'showstringspaces'
  showstringspaces=false,          % underline spaces within strings only
  showtabs=false,                  % show tabs within strings adding particular underscores
  stepnumber=2,                    % the step between two line-numbers. If it's 1, each line will be numbered
  stringstyle=\color{mymauve},     % string literal style
  tabsize=2,	                   % sets default tabsize to 2 spaces
  title=\lstname                   % show the filename of files included with \lstinputlisting; also try caption instead of title
}


\begin{document}
\title{dokidoki Manual}
\author{kesumu}
\date{March 2016}
\maketitle

\chapter{Overview}

dokidoki aimed at being a AVG development tool based on unity3d. It could improve the efficiency to develop AVG games and also could implement the one-time-code-multiple-platform-run based on the features of unity3d.

\section{Features}

\subsection{Open source}

dokidoki project is open source on github.

\subsection{More clear logic for AVG game}
More clear logic. In the game built by this tool, there exists only 3 main parts: WorldControl, World (contains characters), UI. WorldControl is used for controling the flow of the game. World is the container of apperence in this game world, and also character stands on the stage named world. UI is for setting of the game and in game operation.

As the game proceeds, only actions which written in scripts are excuated.

\subsection{Easy to write script grammer}

To be easy to use, we try to make the script of AVG game is as muck as easy to write, just like you are writting Screenplay.

\subsection{Write once, run on all platforms}

Yes, because this tool based on unity3d, it could run on any device the unity3d supports.
\chapter{Installation}

First go to \href{https://github.com/kesumu/dokidoki}{dokidoki} project page on github to download all related project parts: dokiUnity(unity package), dokiScriptCompiler(compile program), dokiScriptSetting(script definition shared by dokiUnity and dokiScriptCompiler).

\section{Develop on Windows}
\section{Develop on Mac OS}
\section{Develop on Linux}
\chapter{Get Started}

\section{Run sample games}

Inside \textbf{sample} folder in dokidoki, there are sample games on multiple platforms. You could run it to see what games dokidoki could build.

\section{Develop my first game}

Here is the step-to-step tutorial to develop a AVG game based on dikidoki. To start development of your game, firstly create a new folder named MyFirstdokidokiProject (In my case C:/\_Projects/MyFirstdokidokiProject).

\subsection{Write my first dokiScript}

Open your Atom, new file, and save it to disk (../MyFirstdokidokiProject/scripts/MyFirstScript.dks). \textbf{scripts} folder name could be any other name, but it is better not to contain space.

Then start to write your dokidoki scripts. Write the script just the same as below. To learn more detail about scripts writing, you should take a look at Chapter 5. Though you don't know much about dokiScript's grammar now. You could still understand the script easily, for that this script language is designed to be intuitive. 

\begin{lstlisting}
world
	background src=ba0 transition=instant;
	bgm src=bg0;
	>There are someone stand beside the school gate.
	>She looks anxious.

Mary
	role type=character name="Mary";
	posture src=po0;
	voice src=mary000 >It is almost the time.

world
	>I find she is there.
	>And I walk towards her slowly.
	>However, she does not notice me.

I
	role type=player name="Jack";
	voice >Hello, Mary.

world
	>I said to her.
	>Heared what I said, she turns happy face immediately.

Mary
	>I am happy to see you again, Jack.
\end{lstlisting}

After wrote this above dokiScript code, save it and close Atom.

\subsection{Create my first dokiUnity project}



\subsection{Build my first game}
\chapter{Architecture}

System is responsible for arranging every thing inside dokidoki system. It contains a world which has lots of characters and one player. These characters take behaviors to interact with each other to progress the game. Behavior is a kind of action. Action is the minimum unit of things that happens in the AVG game, such as the position changing of a character, a sentence the character spoke. Effect is another kind of action such as the start of playing a sound, the start of the video and so on.

\section{Overview}

\section{dokiScriptSetting}

\section{dokiScriptCompiler}

\section{dokiUnity}
\chapter{dokiScript}

dokiScripts screenplay written by AVG game developer, and script screenplay file would be compiled into a list of actions.

\section{Example}

Here give an example.

\begin{lstlisting}
world
video src=video0;
bgm src=bgm0 mode=loop;
background src=background0 transition=instant;
weather type=snow level=0.2;
>??????????
>????????????
sound src=sound0;
>????????


dokiChan
role type=character name = "??"; 
move position=center transition=instant; 
posture src=posture0; 
voice src=voice001 >>??????
voice src=voice002 >>???????
voice src=voice003 >>?????????

["????????????????"(option011, sample1) | "???????????????"(option012, sample1)] 


<option011>


world
>>?????????

I
role type=player;
voice >>??????????????

dokiChan
move position=(0.45, 0.0, 0.0) transition=instant; 
voice src=voice004 >>??????????

I
voice >>????????

(sample2)

<option012>
I
>>????????????????????? 

world
>>??????????????????? 
weather type=sunny; 
>>????????

(sample3)
\end{lstlisting}

\section{Grammar}

\begin{center}
	\begin{longtable}{| C{1cm} | L{8cm} | L{3cm} |}
		\hline
		& Example & Description \\ \hline
		Doki &
		&
		Mostly, one script file \\ \hline
		Flag & 
		\begin{lstlisting}
		[????????????????(option011, sample1) | ???????????????(option012, sample1)]
		\end{lstlisting} & 
		Flag with options to choose \\ \hline
		Option & 
		\begin{lstlisting}
		<option011>
		\end{lstlisting} &  
		Option that Flags should jump here when this option is chosen \\ \hline
		Lock &
		\begin{lstlisting}
		{ xxx }
		\end{lstlisting} & 
		For some performance, disable players? operations \\ \hline
		Block &
		\begin{lstlisting}
		world
		video src=video0;
		bgm src=bgm0 mode=loop;
		backgound src=background0 transition=instant;
		weather type=snow level=0.2;
		??????????>
		????????????>
		sound src=sound0;
		????????>
		\end{lstlisting} & 
		One block of script code, start with one focused Object, maybe world or characterId (here means those lines code would focus on the world) \\ \hline
		Action &
		\begin{lstlisting}
		bgm src=bgm0 mode=loop
		\end{lstlisting} & 
		Action, focused on current \\ \hline
		Tag &
		\begin{lstlisting}
		bgm
		\end{lstlisting} & 
		Action tags \\ \hline
		Key=Value &
		\begin{lstlisting}
		src=bgm0 mode=loop
		\end{lstlisting} & 
		Parameters for current action \\ \hline
		Key &
		\begin{lstlisting}
		src	
		\end{lstlisting} & 
		Attributes key \\ \hline
		Value &
		\begin{lstlisting}
		bgm0
		\end{lstlisting} & 
		Attributes value\\ \hline
		Text &
		\begin{lstlisting}
		??????????
		\end{lstlisting} & 
		\\ \hline
	\end{longtable}
\end{center}


\chapter{dokiUnity}


\chapter{dokiBattle}



\end{document}