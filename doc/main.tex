\documentclass[a4paper]{book}

\usepackage{layouts}
\usepackage{subfiles}
\usepackage{listings}
\usepackage{color}
\usepackage{longtable}

\usepackage{array}
\newcolumntype{L}[1]{>{\raggedright\let\newline\\\arraybackslash\hspace{0pt}}m{#1}}
\newcolumntype{C}[1]{>{\centering\let\newline\\\arraybackslash\hspace{0pt}}m{#1}}
\newcolumntype{R}[1]{>{\raggedleft\let\newline\\\arraybackslash\hspace{0pt}}m{#1}}


\definecolor{mygreen}{rgb}{0,0.6,0}
\definecolor{mygray}{rgb}{0.5,0.5,0.5}
\definecolor{mymauve}{rgb}{0.58,0,0.82}

\lstset{ %
  backgroundcolor=\color{white},   % choose the background color; you must add \usepackage{color} or \usepackage{xcolor}
  basicstyle=\footnotesize,        % the size of the fonts that are used for the code
  breakatwhitespace=false,         % sets if automatic breaks should only happen at whitespace
  breaklines=true,                 % sets automatic line breaking
  captionpos=b,                    % sets the caption-position to bottom
  commentstyle=\color{mygreen},    % comment style
  deletekeywords={...},            % if you want to delete keywords from the given language
  escapeinside={\%*}{*)},          % if you want to add LaTeX within your code
  extendedchars=true,              % lets you use non-ASCII characters; for 8-bits encodings only, does not work with UTF-8
  frame=single,	                   % adds a frame around the code
  keepspaces=true,                 % keeps spaces in text, useful for keeping indentation of code (possibly needs columns=flexible)
  keywordstyle=\color{blue},       % keyword style
  language=xml,                 % the language of the code
  otherkeywords={*,...},           % if you want to add more keywords to the set
  numbers=none,                    % where to put the line-numbers; possible values are (none, left, right)
  numbersep=5pt,                   % how far the line-numbers are from the code
  numberstyle=\tiny\color{mygray}, % the style that is used for the line-numbers
  rulecolor=\color{black},         % if not set, the frame-color may be changed on line-breaks within not-black text (e.g. comments (green here))
  showspaces=false,                % show spaces everywhere adding particular underscores; it overrides 'showstringspaces'
  showstringspaces=false,          % underline spaces within strings only
  showtabs=false,                  % show tabs within strings adding particular underscores
  stepnumber=2,                    % the step between two line-numbers. If it's 1, each line will be numbered
  stringstyle=\color{mymauve},     % string literal style
  tabsize=2,	                   % sets default tabsize to 2 spaces
  title=\lstname                   % show the filename of files included with \lstinputlisting; also try caption instead of title
}


\begin{document}
\title{dokidoki Manual}
\author{kesumu}
\date{May 2016}
\maketitle


\chapter{Introduction}

dokidoki aimed at being a AVG development tool based on unity3d. It could improve the efficiency to develop AVG games and also could implement the one-time-code-multiple-platform-run based on the features of unity3d.
\chapter{Related Projects}
\section{Kirikiri}
\section{NScript}
\chapter{System Architecture}
\section{System}

System class is responsible for arranging every thing inside dokidoki system. It contains a world which has lots of characters and one player. These characters take behaviors to interact with each other to progress the game. Behavior is a kind of action. Action is the minimum unit of things that happens in the AVG game, such as the position changing of a character, a sentence the character spoke. Effect is another kind of action such as the start of playing a sound, the start of the video and so on.

\section{Script}

Scripts class is responsible to recognize the script written by AVG game developer, and then compile it into action sequences for System to conduct.

\subsection{Example}

Here give an example.

\begin{lstlisting}
world
	video src=video0;
	bgm src=bgm0 mode=loop;
	background src=background0 transition=instant;
	weather type=snow level=0.2;
	>??????????
	>????????????
	sound src=sound0;
	>????????


dokiChan
	role type=character name = "??"; 
	move position=center transition=instant; 
	posture src=posture0; 
	voice src=voice001 >>??????
	voice src=voice002 >>???????
	voice src=voice003 >>?????????

["????????????????"(option011, sample1) | "???????????????"(option012, sample1)] 


<option011>


world
	>>?????????

I
	role type=player;
	voice >>??????????????

dokiChan
	move position=(0.45, 0.0, 0.0) transition=instant; 
	voice src=voice004 >>??????????

I
	voice >>????????

(sample2)

<option012>
I
	>>????????????????????? 

world
	>>??????????????????? 
	weather type=sunny; 
	>>????????

(sample3)
\end{lstlisting}

\subsection{Grammar}

\begin{center}
\begin{longtable}{| C{1cm} | L{8cm} | L{3cm} |}
  \hline
     & Example & Description \\ \hline
  Doki &
  &
  Mostly, one script file \\ \hline
  Flag & 
  \begin{lstlisting}
  [????????????????(option011, sample1) | ???????????????(option012, sample1)]
  \end{lstlisting} & 
  Flag with options to choose \\ \hline
  Option & 
  \begin{lstlisting}
  <option011>
  \end{lstlisting} &  
  Option that Flags should jump here when this option is chosen \\ \hline
  Lock &
  \begin{lstlisting}
  { xxx }
  \end{lstlisting} & 
  For some performance, disable players? operations \\ \hline
  Block &
  \begin{lstlisting}
  world
	video src=video0;
	bgm src=bgm0 mode=loop;
	backgound src=background0 transition=instant;
	weather type=snow level=0.2;
	??????????>
	????????????>
	sound src=sound0;
	????????>
  \end{lstlisting} & 
  One block of script code, start with one focused Object, maybe world or characterId (here means those lines code would focus on the world) \\ \hline
  Action &
  \begin{lstlisting}
  bgm src=bgm0 mode=loop
  \end{lstlisting} & 
  Action, focused on current \\ \hline
  Tag &
  \begin{lstlisting}
  bgm
  \end{lstlisting} & 
  Action tags \\ \hline
  Key=Value &
  \begin{lstlisting}
  src=bgm0 mode=loop
  \end{lstlisting} & 
  Parameters for current action \\ \hline
  Key &
  \begin{lstlisting}
  src	
  \end{lstlisting} & 
  Attributes key \\ \hline
  Value &
  \begin{lstlisting}
  bgm0
  \end{lstlisting} & 
  Attributes value\\ \hline
  Text &
  \begin{lstlisting}
  ??????????
  \end{lstlisting} & 
  \\ \hline
\end{longtable}
\end{center}

vfdsg



\end{document}